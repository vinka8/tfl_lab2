\documentclass[a4paper, 14pt]{extarticle}

% Поля
%--------------------------------------
\usepackage{geometry}
\geometry{a4paper,tmargin=2cm,bmargin=2cm,lmargin=3cm,rmargin=1cm}
%--------------------------------------


%Russian-specific packages
%--------------------------------------
\usepackage[T2A]{fontenc}
\usepackage[utf8]{inputenc} 
\usepackage[english, main=russian]{babel}
%--------------------------------------

\usepackage{textcomp}

% Красная строка
%--------------------------------------
\usepackage{indentfirst}               
%--------------------------------------             


%Graphics
%--------------------------------------
\usepackage{graphicx}
\graphicspath{ {./images/} }
\usepackage{wrapfig}
%--------------------------------------

% Полуторный интервал
%--------------------------------------
\linespread{1.3}                    
%--------------------------------------

%Выравнивание и переносы
%--------------------------------------
% Избавляемся от переполнений
\sloppy
% Запрещаем разрыв страницы после первой строки абзаца
\clubpenalty=10000
% Запрещаем разрыв страницы после последней строки абзаца
\widowpenalty=10000
%--------------------------------------

%Списки
\usepackage{enumitem}
\usepackage{array}

%Подписи
\usepackage{caption} 

%Гиперссылки
\usepackage{hyperref}

\hypersetup {
	unicode=true
}

%Рисунки
%--------------------------------------
\DeclareCaptionLabelSeparator*{emdash}{~--- }
\captionsetup[figure]{labelsep=emdash,font=onehalfspacing,position=bottom}
%--------------------------------------

\usepackage{tempora}

%Листинги
%--------------------------------------
\usepackage{listings}
\usepackage{xcolor}

\definecolor{codegreen}{rgb}{0,0.6,0}
\definecolor{codegray}{rgb}{0.5,0.5,0.5}
\definecolor{codepurple}{rgb}{0.58,0,0.82}

\lstset{   
    commentstyle=\color{codegreen},
    keywordstyle=\color{magenta},
    numberstyle=\tiny\color{codegray},
    stringstyle=\color{codepurple},
    basicstyle=\ttfamily\footnotesize,
    breakatwhitespace=false,                                                                 
    breaklines=true,                 
    captionpos=b,                    
    keepspaces=true,                 
    numbers=left,                    
    numbersep=5pt,                                  
    showspaces=false,                
    showstringspaces=false,
    showtabs=false,                  
    tabsize=2
}
%--------------------------------------

%%% Математические пакеты %%%
%--------------------------------------
\usepackage{amsthm,amsfonts,amsmath,amssymb,amscd}  % Математические дополнения от AMS
\usepackage{mathtools}                              % Добавляет окружение multlined
\usepackage[perpage]{footmisc}
%--------------------------------------

%--------------------------------------
%			НАЧАЛО ДОКУМЕНТА
%--------------------------------------

\begin{document}

%--------------------------------------
%			ТИТУЛЬНЫЙ ЛИСТ
%--------------------------------------
\begin{titlepage}
\thispagestyle{empty}
\newpage


%Шапка титульного листа
%--------------------------------------
\vspace*{-60pt}
\hspace{-65pt}
\begin{minipage}{0.3\textwidth}
\hspace*{-20pt}\centering
\includegraphics[width=\textwidth]{emblem}
\end{minipage}
\begin{minipage}{0.67\textwidth}\small \textbf{
\vspace*{-0.7ex}
\hspace*{-6pt}\centerline{Министерство науки и высшего образования Российской Федерации}
\vspace*{-0.7ex}
\centerline{Федеральное государственное бюджетное образовательное учреждение }
\vspace*{-0.7ex}
\centerline{высшего образования}
\vspace*{-0.7ex}
\centerline{<<Московский государственный технический университет}
\vspace*{-0.7ex}
\centerline{имени Н.Э. Баумана}
\vspace*{-0.7ex}
\centerline{(национальный исследовательский университет)>>}
\vspace*{-0.7ex}
\centerline{(МГТУ им. Н.Э. Баумана)}}
\end{minipage}
%--------------------------------------

%Полосы
%--------------------------------------
\vspace{-25pt}
\hspace{-35pt}\rule{\textwidth}{2.3pt}

\vspace*{-20.3pt}
\hspace{-35pt}\rule{\textwidth}{0.4pt}
%--------------------------------------

\vspace{1.5ex}
\hspace{-35pt} \noindent \small ФАКУЛЬТЕТ\hspace{80pt} <<Информатика и системы управления>>

\vspace*{-16pt}
\hspace{47pt}\rule{0.83\textwidth}{0.4pt}

\vspace{0.5ex}
\hspace{-35pt} \noindent \small КАФЕДРА\hspace{50pt} <<Теоретическая информатика и компьютерные технологии>>

\vspace*{-16pt}
\hspace{30pt}\rule{0.866\textwidth}{0.4pt}
  
\vspace{11em}

\begin{center}
\Large {\bf Лабораторная работа № 2} \\ 
\large {\bf по курсу <<Теория формальных языков>>} \\
\end{center}\normalsize

\vspace{8em}


\begin{flushright}
  {Студент группы ИУ9-51Б Винокурова Е. С. \hspace*{15pt}\\ 
  \vspace{2ex}
  Преподаватель Непейвода А.Н.\hspace*{15pt}}
\end{flushright}

\bigskip

\vfill
 

\begin{center}
\textsl{Москва 2025}
\end{center}
\end{titlepage}
%--------------------------------------
%		КОНЕЦ ТИТУЛЬНОГО ЛИСТА
%--------------------------------------

\renewcommand{\ttdefault}{pcr}

\setlength{\tabcolsep}{3pt}
\newpage
\setcounter{page}{2}
\tableofcontents
\newpage

\section{Задача}\label{Sect::task}
По имеющемуся академическому регулярному выражению построить:

\begin{itemize}
\item Минимальный ДКА, распознающий его язык (минимальность обосновать таблицей классов эквивалентности)

\item Возможно малый НКА, распознающий его язык. Возможно малый переключающийся (с конъюнкцией) КА, распознающий его язык. Частично обосновать таблицами множеств классов эквивалентности.

\item Расширенное регулярное выражение, распознающее тот же язык. В расширенном выражении можно использовать:

\begin{itemize}
\item wildcard-операцию . для замены произвольного символа алфавита;
\item положительную итерацию $\tau^+$ и опцию $\tau?$. $\tau^+ = \tau\tau^*$, $\tau? = (\tau|\varepsilon)$;
\item операции предпросмотра $\tau_0(?= \tau_1)\tau_2 \equiv \tau_0((\tau_1.*) \cap \tau_2)$ и ретроспективной проверки $\tau_0(?<= \tau_1)\tau_2 \equiv (\tau_0 \cap (\tau_1.*))\tau_2$, а также их отрицательные версии $\tau_0(?! \tau_1)\tau_2 \equiv \tau_0((\tau_1.*) \cap \tau_2)$ и $\tau_0(?<! \tau_1)\tau_2 \equiv (\tau_0 \cap (\tau_1.*))\tau_2$

\item классы букв $[c_1 \ldots c_k] \equiv (c_1|c_2|\ldots|c_k)$ и их дополнения $[$\^{}$c_1 \ldots c_k]$.
\item (обязательно) маркеры начала и конца выражения \^{} и \$.
\end{itemize}
\end{itemize}

Провести автоматическое тестирование предполагаемой эквивалентности построенных распознавателей. Тем самым необходимо построить алгоритмы, определяющие принадлежность слова языку академического регулярного выражения, ДКА, НКА и ПКА.

Требуется только фаза-тестирование эквивалентности: строится случайное слово $\omega$ и проверяется, принадлежит ли он языкам регулярного выражения, ДКА, НКА и ПКА согласованно.

\textbf{Вариант 4}
\[
((aa | ab | ac)^*(bc | cc | ac)bb | abacc^*)^*
\]
\newpage



\section{Минимальный ДКА}
На основе академического регулярного выражения был построен недетерминированный конечный автомат, который затем детерминизирован и минимизирован; в результате получен минимальный детерминированный конечный автомат, распознающий данный язык. Минимальность ДКА доказана таблицей эквивалентности состояний.

\begin{center}
\includegraphics[width=170mm]{1}
\end{center}

Изображение этого ДКА находится в файле DKA.png

\subsection{Обоснование минимальности ДКА}
\begin{flushleft}
\begin{tabular}{| m{1.5cm} | m{1cm} | m{1cm} | m{1cm} | m{1cm} | m{1cm} | m{1cm} | m{1cm} | m{1cm} | m{1.2cm} |} 
\multicolumn{10}{c}{Таблица префиксов и суффиксов для ДКА} \\
  \hline
   - & $\varepsilon$ & $b$ & $bb$ & $c$ & $ac$ & $bcbb$ & $cbb$ & $bac$ & $accbb$ \\ 
  \hline
  $abac$ & 1 & 0 & 1 & 1 & 0 & 1 & 0 & 0 & 0 & 
  \hline
  $abaccc$ & 1 & 0 & 1 & 1 & 0 & 1 & 1 & 0 & 0 & 
  \hline
  $abacc$ & 1 & 0 & 0 & 1 & 0 & 1 & 1 & 0 & 0 & 
  \hline
  $\varepsilon$ & 1 & 0 & 0 & 0 & 0 & 1 & 0 & 0 & 0 &  
  \hline
  $acb$ & 0 & 1 & 0 & 0 & 0 & 0 & 1 & 0 & 0 & 
  \hline
  $acbcb$ & 0 & 1 & 0 & 0 & 0 & 0 & 0 & 0 & 0 & 
  \hline
  $aba$ & 0 & 0 & 0 & 1 & 0 & 0 & 1 & 0 & 1 &  
  \hline
  $ac$ & 0 & 0 & 1 & 0 & 0 & 1 & 0 & 0 & 0 &  
  \hline
  $acbc$ & 0 & 0 & 1 & 0 & 0 & 0 & 0 & 0 & 0 &  
  \hline
  $ab$ & 0 & 0 & 0 & 0 & 1 & 1 & 0 & 0 & 0 &  
  \hline
  $a$ & 0 & 0 & 0 & 0 & 0 & 0 & 1 & 1 & 1 &  
  \hline
  $aaa$ & 0 & 0 & 0 & 0 & 0 & 0 & 1 & 0 & 1 &  
  \hline
  $aab$ & 0 & 0 & 0 & 0 & 0 & 0 & 1 & 0 & 0 &  
  \hline
  $T$ & 0 & 0 & 0 & 0 & 0 & 0 & 0 & 0 & 0 &  
  \hline
  $aa$ & 0 & 0 & 0 & 0 & 0 & 1 & 0 & 0 & 0 &  
  \hline
\end{tabular}

\vspace{0.7cm}

\end{flushleft}
Построенный ДКА является минимальным, поскольку все его состояния различимы. Это подтверждается таблицей префиксов и суффиксов, так как все строки в ней различны.

\section{Возможно малый НКА}
На основе академического регулярного выражения с использованием линеаризации был построен, а затем частично минимизирован недетерминированный конечный автомат распознающий его язык.

\begin{center}
\includegraphics[width=170mm]{NKA}
\end{center}

Изображение этого НКА находится в файле NKA.png

\subsection{Частичное обоснование минимальности НКА}
\begin{flushleft}
\begin{tabular}{| m{1.5cm} | m{1cm} | m{1cm} | m{1cm} | m{1cm} | m{1cm} | m{1cm} | m{1cm} |} 
\multicolumn{8}{c}{Таблица префиксов и суффиксов для НКА} \\
  \hline
   - & $bac$ & $ac$ & $b$ & $c$ & $\varepsilon$ & $bb$ & $cbb$ \\ 
  \hline
  $a$ & 1 & 0 & 0 & 0 & 0 & 0 & 1 & 
  \hline
  $ab$ & 0 & 1 & 0 & 0 & 0 & 0 & 0 & 
  \hline
  $bcb$ & 0 & 0 & 1 & 0 & 0 & 0 & 0 & 
  \hline
  $aba$ & 0 & 0 & 0 & 1 & 0 & 0 & 1 &  
  \hline
  $\varepsilon$ & 0 & 0 & 0 & 0 & 1 & 0 & 0 & 
  \hline
  $bc$ & 0 & 0 & 0 & 0 & 0 & 1 & 0 & 
  \hline
  $b$ & 0 & 0 & 0 & 0 & 0 & 0 & 1 &  
  \hline
\end{tabular}

\vspace{0.7cm}

\end{flushleft}

\section{Возможно малый ПКА}
На основе академического регулярного выражения был построен переключающийся конечный автомат распознающий его язык.

Каждая ветка ПКА распознает одно из регулярных выражений $L_1$ и $L_2$, которые в пересечении дают исходную регулярку.

$L_1 = ((aa|ab|ac)^*(bc|cc|ac)bb | aba(a|c)^*c)^*$

$L_2 = (((a|b)(a|b|c))^*(a|b|c)(a|b|c)bb | abacc^*)^*$

\begin{center}
\includegraphics[width=170mm]{PKA3}
\end{center}

Изображение этого ПКА находится в файле PKA.png

\subsection{частичное обоснование минимальности ПКА}
\begin{flushleft}
\begin{tabular}{| m{1.5cm} | m{1cm} | m{1cm} | m{1cm} | m{1cm} | m{1cm} |} 
\multicolumn{6}{c}{Таблица префиксов и суффиксов для ПКА} \\
  \hline
   - & $b$ & $bb$ & $bcbb$ & $c$ & $cbb$ \\ 
  \hline
  $abaccc$ & 0 & 1 & 1 & 1 & 1 & 
  \hline
  $abacc$ & 0 & 0 & 1 & 1 & 1 & 
  \hline
  $aba$ & 0 & 0 & 0 & 1 & 1 & 
  \hline
  $aaa$ & 0 & 0 & 0 & 0 & 1 & 
  \hline
  $aa$ & 0 & 0 & 1 & 0 & 0 & 
  \hline
\end{tabular}

\vspace{0.7cm}

\end{flushleft}

\section{Расширенное регулярное выражение}
Исходное регулярное выражение
\[
((aa | ab | ac)^*(bc | cc | ac)bb | abacc^*)^*
\]
описывает произвольную (возможно пустую) последовательность блоков двух форм.

В первой форме блока подвыражение $(aa | ab | ac)$ представляет собой все пары символов, начинающиеся с $a$ и продолжающиеся любой буквой из множества $\{a,b,c\}$. Это эквивалентно записи $(aa | ab | ac) \equiv a[abc]$

Повторение данной группы даёт
$(aa | ab | ac)^* \equiv (a[abc])^*$

Аналогично, подвыражение 
$(bc | cc | ac)$ всегда имеет вид $xc$, где  $x \in \{a,b,c\}$, то есть
$(bc | cc | ac) \equiv [abc]c$

Таким образом, первая форма блока становится $(a[abc])^*[abc]cbb$

Вторая форма блока имеет вид $abacc^*$. Поскольку после обязательного символа $c$ следует $c^*$, общее число букв $c$ в конце не менее одного, следовательно,
 $abacc^* \equiv abac+$.


Объединяя обе формы, а также сохраняя внешнюю звезду, получаем эквивалентное расширенное регулярное выражение:
\[
^\land((a[abc])^*[abc]cbb | abac^+)^*\$.
\]

\section{Фазз-тестирование}
Для проверки эквивалентности построенных распознавателей было написано автоматическое тестирование.

Программа генерирует случайные слова над алфавитом {a,b,c} как полностью случайные, так и случайные, которые соответствуют регулярному выражению.
Для каждого слова выполняется проверка принадлежности языку с использованием четырёх методов: регулярного выражения, минимального ДКА, построенного НКА, переключающегося конечного автомата.

Код программы представлен в test.cpp.

\end{document}
